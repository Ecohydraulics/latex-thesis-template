\chapter{State-of-the-Art}
%\chapter{Theoretische Grundlagen}
\label{ch:soa}

% This chapter can be also called Literature Review

Check what others have done, which is relevant to your research question and to provide evidence for testing the hypotheses defined in \autoref{sec:rq}.

For coherence: note that chapter titles should be \textit{Camel Cased}, while everything else is \textit{Sentence cased}, including all kinds of section names.

\section{Previous works}
\label{sec:prevworks}
	
% For referencs in author-year style, use:
As explained in \citeA{negreiros2024database}. This statement can also be supported by multiple sources \cite{schwindt2019lifespan, schwindt2023bayesian, schwindt2023bedload, scolari2025hydromorphodynamic}.

	
\section{Types of something and image manipulation}
\label{sec:typesome}

Do \citeA{kundu2008fluid} talk about Lagrangian and Eulerian concepts visualized in \autoref{fig:example}?

All figures and images should be stored in the \texttt{images} folder, so that in the \inlinecode{figure} environment, typing the image filename without the extension (JPG, PNG, etc.) in the \inlinecode{\includegraphics} command is sufficient.

Note that saving graphics in JPG format is usually the best option, and PNG is only useful for charts with low pixel coverage.

JPG is uitable for photographs and complex images where a smaller file size is important. It uses lossy compression, which reduces file size substantially. When using the JPG format for a thesis report, be sure to save at 300 dpi (dots per inch).

PNG is best used when you need lossless compression (heavy files!) and transparency support. It is ideal for charts with sharp edges, text, or areas of \textbf{solid} color, as it preserves detail without introducing compression artifacts.

GNU Image Manipulation Program, GIMP, is a powerful open-source image editing software suitable for a wide range of tasks, including photo retouching, image composition, and graphic design. It supports various file formats, including PNG and JPG, and provides advanced tools such as layer management, customizable brushes, and color correction features. GIMP is an excellent free alternative to commercial software, making it a great choice for students who ambition high-quality image editing capabilities without licensing costs.


\begin{figure}[htp]
	\begin{center}
		\begin{minipage}{\textwidth}
			\centering
			\resizebox*{.9\textwidth}{!}{\includegraphics{example}}
			\caption[An example figure.]{An example figure that visually tries to integrate Lagrangian and Eulerian concepts.}
			\label{fig:example}
		\end{minipage}
	\end{center}
\end{figure}

		
\subsection{A subsection with table}
\label{subsec:somesome}
		
As the \autoref{tab:typesomething} shows, this text has to introduce the thing before the table lists the use of the thing.
		
		
\begin{table}[hb] % try [h]ere, then [b]ottom to make sure the table is mentioned in the text before it is shown
	\centering
	\caption{Captions of tables should be positioned above the table, while figure captions should be in the bottom}
	\begin{tabular}{ll}
		\hline
		\textbf{Thing} & \textbf{Use} \\
		\hline
		something & something \\
		something & something \\
		something & something \\
		\hline
	\end{tabular}
	% the label goes below the tabular env
	\label{tab:typesomething}
\end{table}
		
\subsection{No subsection goes alone}
\label{subsec:somenoth}

And it should also have some text.


\section{Something statistics with an equation}
\label{sec:somestatistics}
	
As shown in \autoref{eq:happiness}
\begin{equation}\label{eq:happiness}
	\mbox{happiness}=\frac{\mbox{EmptyCup}+\mbox{FavoriteDrink}}{\mbox{EmptyCups}}
\end{equation}	
	
\section{A section header}
\label{sec:some-ref}
	
\subsection{The logic underlying something in a DEF BOX}
\label{subsec:logicunderlying}
    	
\defbox{The thing}{
	This is the definition of the thing.
}
    	
\subsection{Concepts and terminology}
\label{subsec:conceptsterm}	
		
\subsubsection*{Something set rules} 

Do not use \inlinecode{\subsubsection} levels. If needed, use  \inlinecode{\subsubsection*} instead. This makes a difference in the table of contents.

\section{Something or nothing?}
\label{sec:someornot}

\subsubsection*{Unnumbered non-sense header?}

\infobox{The note}{
	Do you really need to do so much numbering?
}
   	
\section{Equations}

Momentum conservation is described by the Navier-Stokes equation  (\autoref{eq:ns}):

\begin{equation} \label{eq:ns}
	\rho~\left[\frac{\partial \mathbf{u}}{\partial t} + \left(\mathbf{u} \cdot \nabla\right) \mathbf{u}\right] = -\nabla p + \rho \nu \nabla^2\mathbf{u} + \rho \mathbf{g} 
\end{equation}

where $t$ is time, $\mathbf{u}$ is flow velocity, $\rho$ is fluid density, $p$ is pressure, $\nu$ is kinematic viscosity, and $\mathbf{g}$ is gravitational acceleration. The viscous force term $\nu \nabla^2 \mathbf{u}$ is not part of the original Euler equations, which describe inviscid flow where viscosity is negligible. However, the simulation of fluid dynamics includes turbulence where viscosity, while often small compared to inertia and pressure forces, plays a critical role in conveying and dissipating kinetic energy \cite{macdonald2007metallurgical,kundu2008fluid}. 
    
