\documentclass[12pt,oneside]{article}
\usepackage{hyperref}
\usepackage[english]{babel}
\begin{document}
	\section{General Instructions} \\
	\subsection{How to start} \\
	To start using this template, you can:
	\begin{enumerate}
		\item Use it in your browser with \href{https://www.overleaf.com/}{Overleaf}
			\begin{itemize}
				\item Download this project as a .zip folder.
				\item Load the .zip folder in Overleaf
				\item Adapt, change it, write, and compile it in any .tex file of your project (Crtl + S)
			\end{itemize}
		\item Use it with your own Tex Editor. We recommend TexStudio.
			\begin{itemize}
				\item Download TexStudio \href{https://www.texstudio.org/}{TexStudio}
				\item Install the necessary packages, which are: babel, microtype, parskip, fourier, fancyhdr, hyperref, apacite, geometry, graphicx, caption, amsmath, gensymb, enumitem. 
				\item Obs: If you try to compile main.tex you'll receive a message suggesting to install the packages already.
				\item Of course you can delete some of these packages if you'll not use them, as well as add more if you need.
			\end{itemize} 
	\end{enumerate}

	\subsection{General Remarks}
	\begin{enumerate}
		\item Pay attention to proofreading in whichever Tex editor you use. Define if you'll use English (UK) or English (USA).

	\end{enumerate}

	\section{Recommended Books on Scientific Writing}
	\begin{enumerate}
		\item Manual for Writers of Research Papers, Theses, and Dissertations, Chicago Style for Students and Researchers (Manual for Writers of Research Papers, Theses and Dissertations)
		\item Science Research Writing For Non-Native Speakers Of English, Imperial College London
	\end{enumerate}
\end{document}